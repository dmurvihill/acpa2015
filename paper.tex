\documentclass[conference]{IEEEtran}
\usepackage{amsmath,amssymb,amsfonts}
\usepackage{url}
\usepackage{subfigure}
\usepackage{booktabs,threeparttable,multirow}

% new math operators
\DeclareMathOperator{\abs}{abs}

% todo command
\usepackage{marginnote}
\newcounter{todocnt}
\newcommand{\Sim}{\textsc{Simon}} 
\setcounter{todocnt}{0}
\newcommand{\todo}[1]{\stepcounter{todocnt}{\tt {[#1]}} \marginpar{{$\blacksquare$ \thetodocnt}}}  
\newcommand{\specialcell}[2][c]{%
  \begin{tabular}[#1]{@{}c@{}}#2\end{tabular}}

\hyphenation{op-tical net-works semi-con-duc-tor}
\IEEEoverridecommandlockouts
\begin{document}

\title{Countermeasures against co-location attempts in the cloud}


\author{\IEEEauthorblockN{Dolan Murvihill, Nathan Wells
}
\IEEEauthorblockA{Worcester Polytechnic Institute, 
Worcester, MA 01609, USA\\
Email: \texttt{\{dm, nhwells\}@wpi.edu}
}}
\maketitle

\begin{abstract}
Recent research has demonstrated that cloud services such as Amazon AWS do not provide perfect isolation~--- that there
  are side channels between virtual machines that reside on the same host.
In particular, it is possible, under some circumstances, to use cache timing information to perform a full AES key
  recovery against a VM sharing the same cache.

Other work has shown that cloud services such as Amazon AWS provision virtual machines using straightforward, often
  predictable algorithms, and others have demonstrated that it is possible to achieve the required co-location to
  perform a co-location attack.
We would like to discern whether Amazon AWS has implemented any countermeasures against co-location attacks since the
first successful cloud co-location six years ago; we will repeat their procedure in an attempt to reproduce their
results. If time permits, we will try to improve on their attack.
\end{abstract}

\section{Motivation}
\todo{Know what you want to do and why that is interesting (maybe with bullet points). But do not write this section until you know what you actually have done so that the motivation fits your work.}



\begin{itemize}
  \item Amazon's EC2 service is very popular and used for by variety of customers, some of which have sensitive data.
  \item This makes VMs on EC2 a worthwhile target.
  \item Ristenpart et al. have demonstrated that it is possible to create a VM co-resident with a target instance and
    that existing side-channel attacks can be employed against co-resident instances on the cloud.
  \item We expect that Amazon has since adjusted their VM location algorithms to make the Ristenpart attack more
    difficult. We would like to confirm that these changes have been made, by repeating the Ristenpart attack.
  \item If time permits, we will attempt to defeat Amazon's new countermeasures, or improve on the reliability of the
    Ristenpart attack, whatever is more appropriate.
\end{itemize}

\section{Background}\label{sec:background}
\todo{You should find and describe related work early on. Know what other people have done.}

\section{Work Description}
We reproduced two parts of the Ristenpart attack: first, a ``network cartography'' project that tried to find
  correlations between AWS instance creation parameters and IP address; and second, a ``covert channel'' approach to
  verifying co-location.
Our goal is to determine whether one of our instances shares a virtual machine with an arbitrary AWS instance.

\subsection{Covert Channel}
Our goal is to be able to discern co-location with an arbitrary instance, but we can begin by attempting to discern
  co-location with another instance under our control.
Ristenpart et al. evaluated co-location by attempting to transmit data across a covert channel between two virtual
  machines --- in their case, by causing seeks and measuring seek times. Our approach is essentially the same.
A ``sender'' process would continually read values from random locations on the disk to send a one, and do nothing to
  send a zero.
A ``receiver'' process would constantly be reading values, and measuring the seek times.
If the processes shared a hard drive (and thus, a host), the receiver process would be expected to observe consistently
  higher seek times while the sender is transmitting ones than while the sender is transmitting zeroes.
Thus, successfully transmitting a message across this secret channel would imply that the virtual machines are
  co-located.

\subsubsection{Verification on Local Machine}
Some calibration measurements were taken to make sure that the covert channel approach would work.
The ``seeker'', a short C program, was adapted from a 2007 Linux Insider article (TODO cite) to provide the seek
  capability and measurement.
The seeker read random locations from the disk as frequently as possible for thirty seconds, then reported the mean
  time required for each reading.
The seeker was run one hundred times on a personal computer belonging to one of the authors, and the mean seek time
  from each sample was recorded.

(TODO results from laptop test)

A similar test was performed on the same personal computer that ran two seekers at once.
One hundred samples were taken from one seeker (the ``reader'').
The other seeker's samples were discarded.

\subsubsection{Calibration on AWS}
Once we were satisfied that the seeker was performing as expected, we used it to calibrate the covert channel on a
  t2.micro AWS instance.
The one-seeker test and two-seeker test were each run one hundred times, and their sampling distributions were
  used to set the mean seek time ranges that corresponded to zero and one.

\subsubsection{Discovery of co-located Virtual Machines}
(TODO \#) (TODO type) instances were set up in the (TODO az) availability zone, spread evenly across two AWS accounts.
Each virtual machine tried to transmit (TODO \#bits) bits, at a bit rate of one per 30 seconds, to each other virtual
  machine.
(TODO \#) virtual machines were able to successfully transmit the message, (TODO \#) of which had one-bit transmission
  errors in one direction, but not the other.
We inferred these (TODO \#) pairs to be co-located.

\section{Results}
\subsection{Covert Channel}
The tests we ran on our personal computer, described in Section~{mth:localcc} provided information that gave us high
  confidence in the validity of the results we measured on AWS. The sampling distribution for these tests are shown in
  Figure~(TODO figure).

Even if the probability distribution of the hard drive seek time was not Gaussian, such a large number of samples were
  taken that the sampling distribution of means can be treated as Gaussian, according to (TODO who?)'s theorem.
We have superimposed Gaussian functions on the histogram in Figure~(TODO figure) to demonstrate our analysis.

The distributions are very narrow and widely separated, giving us high confidence in the accuracy of our message
  transmission.
The sampling distribution of mean seek time in the single-seeker test (transmitting zeroes) was centered on (TODO \#)
  (TODO milliseconds), with a standard deviation of (TODO \#)(TODO milliseconds).
For the double-seeker test (transmitting ones), the center was (TODO milliseconds) with a standard deviation of (TODO
  milliseconds).
Based on these calibration figures, we could read a mean seek time between (TODO milliseconds) and (TODO milliseconds)
  as a zero, and a mean seek time between (TODO milliseconds) and (TODO milliseconds), and accomplish an error rate of
  (TODO error rate) percent.

Our first calibration test on AWS produced an unexpected result; the first \num{53} samples of the one-seeker test gave
  a mean seek time between \SI{0.29}{ms} and \SI{0.31}{ms}, while the last \num{46} were all over \SI{36}{ms}.
Sample \num{54} gave a mean of \SI{0.81}{ms}, indicating that toward the end of that sample, some condition changed
  that affected the seek time of the VM.
The two-seeker test, which ran after the one-seeker test, showed sample means in the \SI{70}{ms} range.

Reducing the number of samples in the test to (TODO number) eliminated the shift as a factor.
However, the test results reported extremely fast seeks, with no discernible difference between the one-seeker test and
  the two-seeker tests.
Details are shown in Figure~(TODO ref).

(TODO figure)

We believe that EC2 instances are now most often backed by Solid State Disks, which do not exhibit the seek-time
  properties we have been trying to exploit.
Instead of requiring an arm to physically rotate into place over a magnetic platter, SSDs provide fast, constant-time
  random access, a property that eliminates the covert channel we were trying to communicate on.
We believe the sudden change in seek times during the first experiment was caused by our virtual machine moving suddenly

The calibration tests we ran on AWS provided a window of (TODO milliseconds)-(TODO milliseconds) for zeroes, and (TODO
  milliseconds)-(TODO milliseconds) for ones, with a similar error rate.
Our VM matching experiment transmitted (TODO \#bits) bits between co-located virtual machines and observed
  (TODO \#errors) errors, a rate of (TODO error rate) percent; the deviation from the expected error rate is not
  significant ($p = (TODO p-value)$).

\section{Conclusion}
\todo{TBD last}



%\bibliographystyle{IEEEtran}

\end{document}
